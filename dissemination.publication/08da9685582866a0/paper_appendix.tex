%CK_APPENDIX={A}%
\onecolumn

\section{Artifact Appendix}
\label{artifact_appendix}

Submission and reviewing methodology: \\
{\em http://cTuning.org/ae/submission-20171101.html}

%%%%%%%%%%%%%%%%%%%%%%%%%%%%%%%%%%%%%%%%%%%%%%%%%%%%%%%%%%%%%%%%%%%%%
\subsection{Abstract}

This Artifact Appendix describes experimental workflow,
artifacts and results from this paper evaluated 
during the 1st reproducible ReQuEST tournament at the ACM ASPLOS'18:

\begin{packed_itemize}
  \item {\bf Latest CK workflow:} \url{https://github.com/dividiti/ck-request-asplos18-mobilenets-armcl-opencl}
  \item {\bf CK results:} \url{https://github.com/ctuning/ck-request-asplos18-results-mobilenets-armcl-opencl}
  \item {\bf Artifact DOI:} \url{https://doi.org/10.1145/3229773}
  \item {\bf ReQuEST submission and reviewing guidelines:} \url{http://cknowledge.org/request-cfp-asplos2018.html} (\cite{request-asplos18})
  \item {\bf ReQuEST goals:} \cite{cm:29db2248aba45e59:0c7348dfbadd5b95}
  \item {\bf ReQuEST workflows:} \url{https://github.com/ctuning/ck-request-asplos18-results}
  \item {\bf ReQuEST scoreboard:} \url{http://cKnowledge.org/request-results}
\end{packed_itemize}

%%%%%%%%%%%%%%%%%%%%%%%%%%%%%%%%%%%%%%%%%%%%%%%%%%%%%%%%%%%%%%%%%%%%%
\subsection{Artifact check-list}

Details: \url{http://cTuning.org/ae/submission_extra.html}

\begin{packed_itemize}
  \item {\bf Algorithm:} image classification
  \item {\bf Program:} MobileNets using Arm Compute Library v17.12+ and TensorFlow 1.7.0+
  \item {\bf Compilation:} GCC v6+ (recommended v7+); Python 2.7+ or 3.4+
  \item {\bf Transformations:}
  \item {\bf Binary:} compiled from source on a target platform
  \item {\bf Data set:} ImageNet 2012 validation (50,000 images)
  \item {\bf Run-time environment:} Linux; OpenCL v1.2+
  \item {\bf Hardware:} Linaro HiKey960, Firefly RK3399 development boards (or similar)
  \item {\bf Run-time state:} CPU and GPU frequencies set to the maximum
  \item {\bf Execution:} 
  \item {\bf Metrics:} total execution time; top1/top5 accuracy over some (all) images from the data set
  \item {\bf Output:} classification result; execution time (throughput/latency) versus accuracy
  \item {\bf Experiments:} automated via CK command line
  \item {\bf How much disk space required (approximately)?} 4..20GB depending on library installations
  \item {\bf How much time is needed to prepare workflow (approximately)?} from 30 minutes to 1 day (if natively building TensorFlow)
  \item {\bf How much time is needed to complete experiments (approximately)?} from 30 minutes to 1 week (if validating accuracy on the full ImageNet validation data set of 50,000 images)
  \item {\bf Publicly available?:} Yes
  \item {\bf Code license(s)?:} BSD 3-clause
  \item {\bf CK workflow framework used?} Yes
  \item {\bf CK workflow URL:} \url{https://github.com/dividiti/ck-request-asplos18-mobilenets-armcl-opencl}
  \item {\bf CK results URL:} \url{https://github.com/ctuning/ck-request-asplos18-results-mobilenets-armcl-opencl}
\end{packed_itemize}

%-------------------------------------------------------------------------------
\subsection{Description}

\subsubsection{How to obtain}

\textbf{NB:} Execute commands prefixed with `\#' sign under the `root' account or with `sudo'.

First, install the Collective Knowledge framework (CK):

\begin{verbatim}
# apt install python python-pip
# pip install ck
\end{verbatim}

Alternatively, you can install the latest CK version from GitHub in the user space:
\begin{verbatim}
$ git clone http://github.com/ctuning/ck
$ export PATH=$PWD/ck/bin:$PATH
$ export PYTHONPATH=$PWD/ck:$PYTHONPATH
\end{verbatim}

\noindent Then, install the required CK repositories:
\begin{verbatim}
$ ck pull repo --url=https://github.com/dividiti/ck-request-asplos18-mobilenets-armcl-opencl
\end{verbatim}

\subsubsection{Install this CK workflow from the ACM Digital Library snapshot}

You can install and test a snapshot of this workflow 
from the ACM Digital Library without interfering with your current CK installation.
Download related file {\tt request-asplos18-artifact-?-ck-workflow.zip}
to a temporary directory, unzip it and then execute the following commands:

\begin{verbatim}
$ . ./prepare_virtual_ck.sh
$ . ./start_virtual_ck.sh
\end{verbatim}

All the required CK repositories will be installed in your current directory.
%You can now proceed with further evaluation as described below.

%-------------------------------------------------------------------------------
\subsubsection{Hardware dependencies}

The new direct convolution kernel optimized for the Arm Mali Bifrost GPU
architecture can be evaluated on a Mali-G71 GPU (e.g.\ found in the HiSilicon
Kirin 960 chip) or Mali-G72 GPU (e.g.\ found in the HiSilicon Kirin 970 chip).
%
The workflow, however, can also be tested on any Arm Mali GPU with the Arm Mali
Midgard GPU architecture (e.g.\ Mali-T880 found in the Rockchip RK3399 chip).

%-------------------------------------------------------------------------------
\subsubsection{Software dependencies}

Software dependencies for CK workflows are automatically installed and rebuilt 
via shared CK packages.\footnote{\url{https://github.com/ctuning/ck/wiki/Shared-packages}}
However, they may still require some native packages:

\paragraph{ArmCL}

\begin{verbatim}
# apt install libblas-dev liblapack-dev libatlas-base-dev
# apt install python-numpy python-scipy
# pip install pillow
\end{verbatim}

\paragraph{TensorFlow}

\begin{verbatim}
# apt install liblapack-dev libatlas-dev
# pip install enum34 mock pillow wheel absl-py scipy
$ ck install ck-env:package:jdk-8u131-universal
$ ck install ck-env:package:tool-bazel-0.11.1-linux
\end{verbatim}

%-------------------------------------------------------------------------------
\subsubsection{Data sets}

\begin{verbatim}
$ ck install package:imagenet-2012-val-min-resized
$ ck install package:imagenet-2012-aux
\end{verbatim}

\paragraph{ArmCL}

\begin{verbatim}
$ ck install package --tags=mobilenet-v1-all,npy
\end{verbatim}

\paragraph{TensorFlow}

\begin{verbatim}
$ ck install package --tags=mobilenet-v1-all,tensorflowmodel,2017_06_14 
\end{verbatim}


%%%%%%%%%%%%%%%%%%%%%%%%%%%%%%%%%%%%%%%%%%%%%%%%%%%%%%%%%%%%%%%%%%%%%
\subsection{Installation}

\paragraph{ArmCL}

\begin{verbatim}
$ ck install package:lib-armcl-opencl-request
$ ck install package:lib-armcl-opencl-18.03 --env.USE_GRAPH=ON --env.USE_NEON=ON
\end{verbatim}

\paragraph{TensorFlow}

\begin{verbatim}
$ ck install package:lib-tensorflow-1.7.0-src-cpu --env.CK_HOST_CPU_NUMBER_OF_PROCESSORS=1
\end{verbatim}


%%%%%%%%%%%%%%%%%%%%%%%%%%%%%%%%%%%%%%%%%%%%%%%%%%%%%%%%%%%%%%%%%%%%%
\subsection{Experiment workflow}

\paragraph{ArmCL}

\begin{verbatim}
$ cd `ck find script:mobilenets-armcl-opencl`
$ python benchmark.py [--repetitions=10]
$ python benchmark.py --accuracy
\end{verbatim}

\paragraph{TensorFlow}

\begin{verbatim}
$ cd `ck find script:mobilenets-tensorflow`
$ python benchmark.py [--repetitions=10]
$ python benchmark.py --accuracy
\end{verbatim}


%%%%%%%%%%%%%%%%%%%%%%%%%%%%%%%%%%%%%%%%%%%%%%%%%%%%%%%%%%%%%%%%%%%%%
\subsection{Evaluation and expected result}

\begin{verbatim}
$ ls `ck find jnotebook:mobilenets-armcl-opencl`
analysis.ipynb
\end{verbatim}

%%%%%%%%%%%%%%%%%%%%%%%%%%%%%%%%%%%%%%%%%%%%%%%%%%%%%%%%%%%%%%%%%%%%%
\subsection{Notes}

We recommend to check {\tt README} from the live CK workflow for the latest
installation and usage instructions: 

\url{https://github.com/dividiti/ck-request-asplos18-mobilenets-armcl-opencl}
